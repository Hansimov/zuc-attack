%# -*- coding: utf-8-unix -*-
\begin{bigabstract} % 8k
This paper studies the technologies used in traditional power analysis of block cipher algorithms and applies them to a famous stream cipher algorithm --- ZUC algorithm. The results of our experiments indicate that ZUC algorithm is also vulnerable to power analysis attacks.

We introduce the basic knowledge of modern cryptography, cryptographic devices and side-channel attacks. 

Modern cryptography is based on rigorous mathematical conclusions and provements, so its security is improved a lot compared to traditional cryptography. One famous principle in modern cryptography is Kerckhoff's principle. The principle suggests that the cipher algorithms should be open and evaluated by the public. This principle breaks people's prejudices on the confidentiality of cipher algorithms and it is proved to be true as time goes on. So the famous cipher algorithms are harder to attack and break with traditional methods of cryptanalysis.

However, the side-channel attacks appeared in the 1990s. This kind of attacks does not aim to discover the bugs or flaws in the cipher algorithms with theoretical provements, but open the gate to attack the cryptographic devices in reality. People using this kind of attacks believe that, although the cipher algorithm is nearly perfect and it is impossible to find their theoretical flaws, the devices running the cipher algorithms are not perfect. Vulnerabilities are common in the real-world implementations of the cipher algorithms.

Since side-channel attacks have large threats to the security of the real-world cryptographic devices, a lot of researchers are attracted and this kind of attacks becomes more and more popular. And the techniques and ideas of side-channel attacks are flourishing and amazing today. From the early timing attack to the latest acoustic attack, from the famous power analysis attack to the less known light attack, side-channel attacks have become the most effective and powerful methods to break all kinds of cryptographic devices.

And one important part of side-channel attacks is the real-world circuits and devices. It is necessary to learn about the design and manufacture of physical devices. The basic elements of cryptographic devices are digital integrated circuits. So the researchers should not only be good at mathematics, but also have a good understanding on microelectronics. This is why side-channel attacks can do some things that traditional cryptanalysis cannot do --- knowledge of different areas is applied.


We discuss the power analysis attacks, and one technique of this kind is differential power analysis, the abbreviation for which is DPA. 

Before attacking the devices with power analysis, we must learn that in the circuits, what is consuming power. The power consumption of logic components is usually divided into static power and dynamic power consumption.

Static power consumption usually comes from the leakage current of transistors, which is relatively low, but it needs to be noted that as the size of a single transistor becomes smaller and smaller, the proportion of leakage current is gradually increased. Of course, static power consumption is negligible in actual attacks.

The dynamic power consumption usually comes from the reversal of the signal, which causes the truncation or conduction of the transistor, which is equivalent to the charge and discharge of the intrinsic and parasitic capacitance of the transistor. Another part of the dynamic power consumption comes from transient short-circuit current. One other thing to consider is the burr produced in the circuit, which often produces high instantaneous power consumption and is related to the data, so special attention needs to be paid. In short, dynamic power consumption is generally a major component of component power consumption, and most of the power consumption we collect is mostly dynamic power consumption.

In the design stage of digital circuits, designers often need to simulate the power consumption of circuits. On the one hand, it is to reduce the power of the circuit as much as possible to improve the market competitiveness of the circuit, on the other hand, to avoid more obvious disadvantageous factors such as burr, affect the basic function of the circuit, and to reduce the information of power leakage as far as possible.

Power analysis usually consists of simple power analysis (SPA) and differential power analysis (DPA).

Simple power analysis usually requires a small number of power curves to reveal useful information in cryptographic devices. Simple power analysis is usually applicable to cryptographic devices with more obvious characteristics of the power curve, such as obvious peaks and valleys, and multiple periodic repetitions. If an attacker has certain preparatory knowledge of the program running in the password device, then it is possible to speculate on what operation of the different paragraphs of the power curve, and more information on the device may be further mastered.

Differential power analysis requires a large number of power traces. The benefits of a large amount of power data are more powerful analysis and attack capabilities, and there is no need to have a detailed understanding of the construction and execution of the device. In general, as long as the algorithm flow of the device runs is sufficient to carry out analysis and attack.

The general process of differential power analysis includes the following steps. Firstly, select the appropriate location of the intermediate value of the algorithm; secondly, collect the actual power consumption curve of the device; thirdly, calculate the theoretical intermediate value according to the algorithm and use the appropriate power model to convert the theoretical middle value to the hypothesis of power consumption; finally, analyze the assumption of the power and the actual power curve, and dig the information needed.


We talk about the ZUC algorithm and propose a scheme to attack it with differential power analysis.

Zu Chongzhi algorithm, also known as ZUC algorithm, is the first international commercial standard cipher algorithm proposed by our country. The proposed ZUC algorithm has made our country have more autonomy in the field of international commercial cryptography, which not only embodies the academic ability of our country in the field of cryptography, but also is an important step for our country to participate in the formulation of international communication standards. Therefore, the study of ZUC algorithm has high practical significance.

Because the password itself is relatively complex, although I have implemented the ZUC algorithm on both hardware and software, and verified the correctness of the implementation, it is unnecessary to explain the details of some implementations, and it is just a rough introduction to the basic composition of the algorithm. Part and general flow. If you want to know more about the algorithm, you can refer to the literature.

As an attacker and researcher, it is important to know all the details of the algorithm and the reason for the design, but it is more important to find the possible problems in the actual implementation of the algorithm. A designer and an attacker consider a different point of view. The designer often has a better view of the overall situation, but it is often unexcavated in detail, and an attacker only needs to pry a point in the whole system to reach the goal.

In the ZUC algorithm, the only unknown information is the initial key, and the other constants and plaintexts are known. Therefore, the purpose of our attack is to get the key information. We choose the output of the right half of the nonlinear function in the first round of the initialization phase as the intermediate value, and try to attack the value of fifth byte of the original keys.

We apply the differential power analysis attack to a real-world device and the results indicate that ZUC algorithm is vulnerable to differential power analysis.

When the number of power consumption curves used by the attack is small, the result is less reliable, and the best key byte guess is not stable enough. Therefore, it is necessary to test the stability of guessing results when attacking. The method of selection is to use more power curves.

When the number of power consumption curves is increasing, the relative correlation coefficient of the correct key byte is obviously higher than that of the wrong key byte.

The reason for this phenomenon is obvious. With the increase of the number of power consumption curves, the difference between the intermediate value produced by the error key byte and the correct key bytes is more obvious, and the contingency is gradually covered by the inevitability. Therefore, the selection of as many power curve bars as possible can generally increase the attack success rate.

With the increase of the number of power curve, the maximum of relative correlation coefficient decreases. In fact, not only the highest relative correlation coefficient is reduced, but all key bytes guess the corresponding relative coefficient, which is decreasing as a whole. 

This is because, although there is a certain relationship between the assumption of the power value (based on the theoretical median and the power model calculated) and the actual power consumption, this connection is not necessarily strongly positive correlation. Therefore, when the number of power consumption curves used by the attack is large, it is assumed that the proportion of the uncorrelated parts between the power consumption value and the actual power consumption will increase.

In conclusion, we prove that the power analysis is effective and powerful to attack the ZUC algorithm. We have developped a general process of problem modeling and data analysis. The experience can also be applied to other stream cipher algorithms and that is the most important result.

\end{bigabstract}