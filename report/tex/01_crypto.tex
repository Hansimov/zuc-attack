%# -*- coding: utf-8-unix -*-
%%==================================================
%% 01_crypto.tex % 4k
%%==================================================

\chapter{现代密码学与旁路攻击}

经过多年的学术研究和工业应用,密码学理论已经日趋系统和完善,各种密码算法广泛应用于各种工业设备,以保障系统和数据的安全。

目前,那些得到广泛使用的密码算法,通常都经过数学上的严格论证,并且经过了大量专家的研究和改进,因而在理论上基本是安全的。然而在现实生活中,这些算法都运行在具体的设备上,因此可能会暴露出各种各样的安全问题,研究者和攻击者可以藉由各种手段,获取密码设备中的秘密信息。

在诸多攻击密码算法和密码设备的手段中,旁路攻击是极为优秀和实用的一类。“旁路”的含义是利用实际密码设备泄露的信息,而不是利用密码算法本身的漏洞。根据时间、成本以及仪器的不同,旁路攻击又可以划分为很多种类,实际的方法也五花八门。在实际应用中,旁路攻击的效果远优于传统的密码学理论分析,因此得到了攻击者和研究人员的青睐。

\section{现代密码学} % 1k

由于目前应用在实际生活中的绝大多数密码学理论都属于现代密码学,因此我们讨论的重点也集中在现代密码学。

现代密码学和古典密码学的一个重要区别是:现代密码学构建在严格的数学论证和完备的系统架构之上。目前普遍使用的算法也都是现代密码算法,并且经过许多分析、攻击和改进,才渐渐保证其理论上的安全性。

信息安全是现代密码学的一个重要应用领域。衡量信息安全的指标包括:保密性、完整性、认证性、不可否认性以及可用性。要达成这些严格的指标,就需要同样严格的现代密码学体系支撑。这些安全指标有些是彼此掣肘的,因此在设计一个密码系统时,需要综合各方面的因素,才能达到预期的目的,保证系统的安全性。

现代密码学遵循一些基本的准则,比如柯克霍夫原则(Kerckhoffs's principle)。该原则指出,系统的安全性不应依赖于具体实现的保密性。信息论的创始人香农(Shannon)也说:“敌人了解系统。”这些原则来源于历史上的经验和教训:几乎所有尝试隐藏算法和系统实现方式本身的做法,最后都失败了。

这些原则促进了通用算法的公开和评审,现实表明,公开的算法经过层层筛选和多次改进,具备更加优良的理论安全性和实际可用性。因此,当前应用广泛的许多密码算法都是公开透明的。我们在进行密码学分析和旁路攻击时,不需要在获取算法细节方面花费不必要的工夫(不过,相应地,这也意味着这些算法本身已经具有极高的安全性,想要分析和攻击它们是一件非常困难的事情)。

划分现代密码学有很多依据,其中一个就是密钥的特征,分为对称密码学和非对称密码学。对称密码学应用的一个典型是 AES 算法(Advanced Encryption Standard,高级加密标准),其特点是加密和解密使用相同的密钥。非对称密码学应用的代表是 RSA 算法(Rivest–Shamir–Adleman),其特点是加密和解密使用不同的密钥。

相比非对称密码学,旁路攻击在对称密码学方面的研究相对多一些。一方面可能是因为对称密码算法加解密采用同一密钥,破解密钥更有意义;另一方面是由于非对称密码算法常常要进行大数运算,速度比对称密码算法要慢很多,因而在对性能要求较高的工业密码设备中,非对称密码算法应用较少,也就造成了实际样本较少,并且研究动力不足。本文涉及的祖冲之算法,也是对称密码算法中的一种。

% 现代密码学 - 百度文库
%   https://wenku.baidu.com/view/7d3a5991daef5ef7ba0d3cce.html

% 现代密码学第1章:现代密码学概论
%   https://wenku.baidu.com/view/f79db2c79ec3d5bbfd0a749c.html

% 流密码与分组密码 - 百度文库
%   https://wenku.baidu.com/view/65316c2300f69e3143323968011ca300a6c3f6d8.html

\section{密码设备} % 1k
数学家和密码学家们常常关注理论层面,试图从根本上提高密码算法的安全性,而工程师则更多地和现实打交道,将这些纸上的东西转化成真正能够工作的软硬件设备。也正是由于这种分工,熟悉密码学理论的数学家和密码学家们难以注意到现实设备在实现时可能存在的安全隐患,而整天从事软硬件设备开发和维护的工程师,也鲜有人能够通晓算法背后的原理,从而留下一些潜在的可以利用的漏洞和瑕疵。

上一小节我们简单提及了密码学的理论部分,因此这一小节我们将讲述一些和实际密码设备相关的知识。了解密码设备的软硬件构成,有助于我们掌握实际设备的特性,以及可能出现的信息泄露点,为我们的旁路攻击打下必要的基础。

密码算法的执行通常涉及到两部分,一部分是数据的保存,另一部分是数据的操作。相应地,密码设备也可以分为两种,一种是保存数据的存储设备,另一种是操作数据的处理设备。

密码设备一般拥有这几个部分:

\begin{itemize}
\item 专用硬件:比如执行特定密码算法的电路
\item 通用硬件:比如控制算法流程的控制电路
\item 存储硬件:用于存储数据和程序
\item 接口:用于规定数据的输入和输出
\end{itemize}

密码设备通常由数字电路实现,不同部分既可以在同一块芯片上完成,也可以分到多个芯片上。我们这里以单芯片的密码设备为例。通常,单芯片数字电路可以通过两种方式实现,一种是 FPGA(Field Programmable Gate Array, 现场可编程门阵列),另一种是 ASIC (Application-Specific Integrated Circuit,专用集成电路)。虽然用不同方式实现的密码设备在运行性能、工作场景以及成本上有所不同,但实现流程却基本相同。

不管是 FPGA 还是 ASIC,其本质都是数字电路。数字电路的基本组成部分是各种逻辑元件。通常有两种逻辑元件,一种是组合元件,实现基本的逻辑功能,比如与、或、非,另一种是时序元件,比如触发器、锁存器以及寄存器。时序元件和组合元件最大的不同是,时序元件的输出不仅取决于当前的输入,还和元件当前的状态有关。

目前,在实现逻辑元件的工艺中,以 CMOS(Complementary Metal Oxide Semiconductor, 互补金属氧化物半导体)最为常见。

\section{旁路攻击} % 2k
传统的密码学分析试图挖掘算法本身存在的问题,抑或是寻找某些数学前提和密码学假设的漏洞,从而在根本上攻破某一密码学算法或系统。一旦从理论上揭露了某种密码学算法或系统的潜在问题,那么暴露出来的问题就是致命的,使用这些算法或系统的设备要么被废弃,要么经受升级和改进,否则就不能消除安全隐患。

然而,正如上文提到的,由于现代密码学主张密码算法设计的公开化和透明化,因此,大多数现在广泛使用的密码算法和系统都经过了严格的论证、长期的研究和持续的改进。一方面,对于使用这些算法和系统的设备来说,安全性得到了极大的提升;另一方面,对于相关领域的研究人员和攻击者来说,使用传统方法破解算法或者是提取算法中的重要信息,变得日益困难。

虽然现代密码学的严格化和公开化的初衷正在于更好地保障信息安全,但是这对攻击者和研究人员无疑也是巨大的挑战:攻击者需要开发更强大的武器和工具,花费更多的时间和成本,才能达成目的;而研究人员也需要付出更多的精力和汗水,学习更多的数学知识,和更多的符号和数字打交道,才能做出点像样的成果,发几篇可能有用的论文。

而旁路分析的出现,则提供了另一种截然不同的思路,并且打开了一扇通往新世界的大门。旁路分析的思路和方法五花八门,各有千秋,一般倾向于认为旁路攻击属于被动型非侵入式攻击,也就是说,这类攻击只是对密码设备的各种攻击方法的一小部分。

以攻击密码设备时使用的接口作为分类依据,一般将攻击的方式划分为三种:侵入式攻击,半侵入式攻击非和侵入式攻击。

下面对这三类攻击作一些简单的介绍和说明:

\begin{itemize}
\item 侵入式攻击:

    能够进行侵入式攻击的人通常能够对密码设备拥有较长时间的物理接触权限,因而可以对设备进行非常仔细的拆卸和分解,能够获得设备详尽的信息和状况。不过,要完成侵入式攻击,付出的代价也相对高,比如需要探测台和激光切片器这样的昂贵仪器和技术。

\item 半侵入式攻击:

半侵入式攻击一般也要对密码设备进行一定程度的拆卸和分解,但是通常不会破坏芯片的钝化膜,不直接接触芯片的表面。半侵入式攻击相比侵入式攻击,在仪器上花费的成本要低一些,但如何选择正确的部位以从设备中读取泄露的信息依旧要耗费大量的时间。比如常见的故障攻击,就属于半侵入式攻击的范畴。

\item 非侵入式攻击:

相比上面两种攻击,非侵入式攻击成本是最低的,并且在攻击完成后,不会有明显的痕迹。狭义上的旁路攻击,就属于非侵入式攻击,只利用设备合法的接口以及泄露的侧信道信息,就能通过数学手段提取出设备中的敏感内容。正因为非侵入式攻击所需的设备以及花费的代价很低,因此比前两种攻击应用更加广泛,威胁也更大。
\end{itemize}







