%# -*- coding: utf-8-unix -*-
%%==================================================
%% 05_conclusion.tex % 4k
%%==================================================

\chapter{总结和展望}

\label{chap:conclusion}

\section{研究总结与创新之处}

本次毕业设计研究了旁路攻击方法在祖冲之算法上的应用,实现了硬件电路,完成了软件分析,得到了预期结果,并且解释了诸多现象的原因。

\vspace*{0.5\baselineskip}

这里总结一下研究的内容和成果。

首先我们了解了现代密码学的基本概念,对现代密码学在信息安全领域的应用有了较为感性和初步的认识;了解了旁路攻击的基本分类和常见手段,以及不同手段的具体适用场景;了解了密码设备的组成部件和基本结构,对数字电路的专业知识做了简单的整理。

然后我们掌握了旁路攻击中最经典的方法——功耗分析攻击,并且介绍了功耗分析攻击的一般流程;了解了数字电路功耗的构成、功耗仿真的方式,以及功耗采集所需的设备与步骤;掌握了功耗分析中最常用的方法——差分功耗分析,详细介绍了差分功耗分析的流程和步骤。

接着我们熟悉了 ZUC 算法的背景和原理,并且讲解了 ZUC 算法的工作方式;完成了 ZUC 算法的硬件电路,采集了设备运行时的功耗,提出了对 ZUC 算法的差分功耗分析方案,并且在软件层面对功耗曲线进行了详细地分析,验证了方案的正确性和可行性。

最后我们讨论了

\vspace*{0.5\baselineskip}



\section{未来展望与后续工作}



