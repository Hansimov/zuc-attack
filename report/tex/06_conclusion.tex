%# -*- coding: utf-8-unix -*-
%%==================================================
%% 05_conclusion.tex % 4k
%%==================================================

\chapter{总结和展望}

\label{chap:conclusion}

\section{研究总结}

本次毕业设计研究了旁路攻击方法在祖冲之算法上的应用,实现了硬件电路,完成了软件分析,得到了预期结果,并且解释了诸多现象的原因。

% 因此在这里总结一下研究的内容和成果。

首先我们了解了现代密码学的基本概念,对现代密码学在信息安全领域的应用有了较为感性和初步的认识;了解了旁路攻击的基本分类和常见手段,以及不同手段的具体适用场景;了解了密码设备的组成部件和基本结构,对数字电路的专业知识做了简单的整理。

然后我们掌握了旁路攻击中最经典的方法——功耗分析攻击,并且介绍了功耗分析攻击的一般流程;了解了数字电路功耗的构成、功耗仿真的方式,以及功耗采集所需的设备与步骤;掌握了功耗分析中最常用的方法——差分功耗分析,详细介绍了差分功耗分析的流程和步骤。

接着我们熟悉了 ZUC 算法的背景和原理,并且讲解了 ZUC 算法的工作方式;完成了 ZUC 算法的硬件电路,采集了设备运行时的功耗,提出了对 ZUC 算法的差分功耗分析方案,并且在软件层面对功耗曲线进行了详细地分析,验证了方案的正确性和可行性。

最后我们展示了实验结果,简单分析了功耗曲线的特征,成功地依据制定的方案完成了攻击,得到了正确的密钥字节。我们还发现了攻击使用的功耗曲线条数对相关系数的有一定的影响,并且根据事实和观察给出了可能的原因和合理的解释。并且根据得到的结果反推出密钥信息的泄露位置,发现实际情况和我们的假设结论是相符的。

\section{未来展望}

尽管最终的实验结果达到了预期的目标,但是还有很多可以改进和深入的地方。

首先是可以对 ZUC 算法的分析方法进行改进。选取文中所述的中间值和功耗模型,只是差分功耗分析中的一个可行方案,而差分功耗分析只是诸多功耗分析中的一种,功耗分析又是很多旁路攻击方法中的一种。比如之后可以尝试其他可能的中间值,或者选用不同的功耗模型。还可以尝试功耗分析之外的其他方法,比如电磁攻击,很可能电磁信息会泄露算法中更多的信息。

然后是可以对分析方法的适用范围进行改进。本次实验仅仅是完成了对 ZUC 算法的攻击,如果能够将这些方法中的核心思路和关键技术抽象出来,把他们应用到更多的序列密码算法电路中,可能会有更高的实用价值。

最后是可以针对 ZUC 算法在实验中暴露的问题给出具体的防护方案。比如可以针对 S 盒进行掩码,掩盖设备运行过程中泄露的信息,或者是在算法的某些地方随机插入空操作,以打乱功耗曲线的时序,让攻击者难以分析出功耗泄露的具体位置。

总而言之,本次毕业设计还有很大的提升空间,如果想要提高和改进实验的结果,就需要我在今后的时间里,学习更多的知识,进行更加深入的研究。所谓学无止境,即使到达百尺竿头,亦需要更进一步。
