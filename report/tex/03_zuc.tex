%# -*- coding: utf-8-unix -*-
%%==================================================
%% 03_zuc.tex % 2.5k
%%==================================================

\chapter{祖冲之算法}
通常的差分功耗分析都是以分组密码作为研究对象的,而本文则以祖冲之密码算法为例来说明,差分功耗分析一样可以攻击序列密码算法。

在实施攻击之前,攻击者一定要对算法本身有充分地研究,这样才能找到可能泄露信息的地方,并加以利用。这一章我们将介绍祖冲之算法的详细知识,试图从算法的流程中寻找可以攻击的位置。

\section{算法背景} % 0.5k
% 《3GPP LTE 国际加密标准 ZUC 算法》
祖冲之算法,又称 ZUC 算法,是我国提出的第一个国际商用标准密码算法。

2004 年,3GPP(3rd Generation Partnership Project,第三代合作伙伴计划)提出了 LTE(Long Term Evolution,长期演进),目的是保证该计划能够继续在电信行业拥有一定的话语权。在 2010 年底,3GPP 被确立为第四代(4G)移动通信标准。

安全性是通信技术中一个非常重要的指标,而密码算法又是保障安全性的一个重要工具。3GPP 之前已经拥有的两个算法是 AES 和 SNOW 3G,而 ZUC 算法则是第三个被纳入标准的算法。ZUC 算法的提出和设计历经了很多挑战,因为商用密码算法的要求非常严苛,既要保证极高的安全性,又要拥有较高地运行性能,还需要在各自环境下都方便实现。中国科学院等单位克服了重重困难,最终研制成功,经由中国通信标准化协会与工信部向 3GPP 组织提交了这一算法,并且经过了行业严格的评审,最终被批准成为 LTE 中的密码算法标准,参与到实际的商业应用。

总之,ZUC 算法的提出,使我国在国际商用密码领域拥有了更多的自主权,既体现了我国在密码学领域的学术能力,又是我国参与制定国际化通信标准的重要一步。因此,研究 ZUC 算法,具有很高的现实意义。

\section{算法流程} % 1k

ZUC 算法分成三层,如下:

\begin{itemize}
    \item \textbf{线性反馈移位寄存器(Linear feedback shift register,LFSR):}LFSR 有两种模式,初始化模式和工作模式。
    \item \textbf{比特重组(Bit Reorgnization, BR):}该层将寄存器中特定位置的数据进行编排然后输出。
    \item \textbf{非线性函数(Nonlinear Function, F):}这一层涉及到一个 S 盒置换,而 S 盒置换是非线性的,因此也是这个算法中非常关键的部分。我们后续的功耗分析攻击,将会重点关注这一部分。
\end{itemize}

\vspace*{\baselineskip}

ZUC 算法的运行过程如下:

\begin{enumerate}
    \item \textbf{初始化阶段:}这一部分包括初始密钥的装载,LFSR 和寄存器的初始化,以及重复 32 轮的打乱操作(每一轮都包含比特重组、非线性函数以及以初始模式运行一次 LFSR)。
    \item \textbf{工作阶段:}这一部分包括一个一次性操作(比特重组、非线性函数以及以工作模式运行一次 LFSR)以及密钥生成阶段(每一次密钥输出都包含比特重组、非线性函数、输出密钥以及以工作模式运行一次 LFSR)。
\end{enumerate}

\vspace*{\baselineskip}

由于密码本身相对比较复杂,尽管我已经在硬件和软件上都实现 ZUC 算法,并且验证了实现的正确性,然而想要解释清楚一些实现中具体的细节,还是太过冗长了,而且毫无必要,因此这里只是粗略地介绍了一下算法的基本组成部分和大致流程。如果想要了解关于算法的更多信息,可以参考文献??。

作为攻击者和研究人员,了解算法的全部细节和设计理由固然重要,但是更加重要的是找到算法在实际实现中可能存在的问题。设计者和攻击者考虑问题的角度是不一样的,设计者往往拥有更好的大局观,但在细节上往往无法挖掘太深,而攻击者只需要撬动整个系统中的某一点就足以达成目标了。

因此我们将在下一小节讨论对 ZUC 算法实施差分功耗攻击的方法。

\vspace*{\baselineskip}

图 \ref{fig:zuc_algo} 展示了 ZUC 算法的过程。

\begin{figure}[htbp]
    \centering
    \includegraphics[height=.5\textheight]{../images/zuc_algo.png}
    \caption{ZUC 算法的流程图}
    \label{fig:zuc_algo}
\end{figure}


\section{对 ZUC 算法的差分功耗攻击} % 1k
\label{sec:zuc_attack}

小节 \ref{sec:dpa} 阐述了差分功耗攻击的基本方法,而本小节将把这个方法应用到具体的算法上。

在 ZUC 算法中,唯一未知的信息就是初始密钥,其他的常量和明文都是已知的。因此我们的攻击目的就是得到密钥的信息。

差分功耗攻击最核心的思想是,假设功耗值和实际功耗值之间是有关联的。而要想假设功耗值尽可能贴合实际功耗值,就需要选择合适的中间值,采用合适的功耗模型。一般而言,中间值通常选择算法中非线性变换的部分,因为如果输入稍有不同,非线性变换的输出就会出现较大的差异,因此,正确的输入和错误的输入产生的差异将比线性变换更加明显,就可以有效地区分出正确的输入和错误的输入。功耗模型通常选择汉明重量模型或者比特模型,为了方便,我们这里采用汉明重量模型。

\vspace*{\baselineskip}

差分功耗分析的第一步是要在硬件上实现 ZUC 算法电路,然后采集其运行时的功耗。这部分和针对其他算法的差分功耗分析没有什么差别,因此不再详细说明。我们重点关注分析 ZUC 算法中的特别之处。

\vspace*{\baselineskip}

图 \ref{fig:zuc_attack} 展示了 ZUC 算法中可以利用的漏洞。在 ZUC 算法初始化阶段的第一轮打乱操作中,非线性函数中左侧寄存器的输出仅和 k9 这个密钥字节相关,右侧寄存器的输出仅和 k5 这个密钥字节相关。

\begin{figure}[htbp]
    \centering
    \includegraphics[height=.5\textheight]{../images/zuc_attack.png}
    \caption{非线性函数中左侧寄存器的输出仅和 k9 相关,右侧寄存器的输出仅和 k5 相关}
    \label{fig:zuc_attack}

\end{figure}

因此,选择初始化阶段第一轮打乱操作中,非线性函数右半部分的输出作为我们的中间值,尝试攻击出 k5 的值。

\vspace*{\baselineskip}

选择这个位置作为中间值有两个原因:

\begin{enumerate}
    \item \textbf{这个位置仅仅和单个密钥字节(k5)相关,和其他的密钥字节没有任何关系。}这意味着我们只需要猜测 $2^8=256$ 中情况,大大减少了工作量。如果选择某个其他的中间值,很可能和多个密钥字节有关联,我们假设关联的密钥字节数为 N,那么我们就需要猜测 $2^{8 \times N}$ 中情况。可以想象,如果关联的密钥字节很多的话,我们需要猜测的可能情况就会爆炸式增长,由于我们只拥有有限的计算资源,因此这是不能接受的。
    \item \textbf{这个位置是非线性变换(S 盒置换)的输出,因而正确猜测和错误猜测之间的差异较为明显。}这一点我们在上面已经提及,因此不再赘述。
\end{enumerate}

\vspace*{\baselineskip}

选择好了中间值,我们就可以通过编程计算出所有密钥猜测(从 0 到 255,也即 16 进制的 00 到 FF)对应的的中间值。

有了中间值之后,我们就可以根据汉明重量模型,计算出理论功耗值(也就是假设功耗值)。然后我们实施相关系数攻击,也即计算假设功耗值和实际功耗值之间的相关系数(在本次实验中,这个值还需要稍微处理一下,得到更加可靠的相对相关系数),然后分析相对相关系数,值最大的即对应最优的密钥猜测。

这就是对 ZUC 算法进行功耗分析攻击的大致流程,我们将在下一小节展示相关的实验结果,并补充一些实现细节。



