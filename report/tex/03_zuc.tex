%# -*- coding: utf-8-unix -*-
%%==================================================
%% 03_zuc.tex % 2.5k
%%==================================================

\chapter{ZUC 算法的软硬件实现}

\label{chap:zuc}

\section{算法背景} % 0.5k
% 《3GPP LTE 国际加密标准 ZUC 算法》
祖冲之算法,又称 ZUC 算法,是我国提出的第一个国际商用标准密码算法。

2004 年,3GPP(3rd Generation Partnership Project,第三代合作伙伴计划)提出了 LTE(Long Term Evolution,长期演进),目的是保证该计划能够继续在电信行业拥有一定的话语权。在 2010 年底,3GPP 被确立为第四代(4G)移动通信标准。 \cite{lte}

安全性是通信技术中一个非常重要的指标,而密码算法又是保障安全性的一个重要工具。3GPP 之前已经拥有的两个算法是 AES 和 SNOW 3G,而 ZUC 算法则是第三个被纳入标准的算法。ZUC 算法的提出和设计历经了很多挑战,因为商用密码算法的要求非常严苛,既要保证极高的安全性,又要拥有较高地运行性能,还需要在各自环境下都方便实现。中国科学院等单位克服了重重困难,最终研制成功,经由中国通信标准化协会与工信部向 3GPP 组织提交了这一算法,并且经过了行业严格的评审,最终被批准成为 LTE 中的密码算法标准,参与到实际的商业应用。 \cite{zuc_test}

总之,ZUC 算法的提出,使我国在国际商用密码领域拥有了更多的自主权,既体现了我国在密码学领域的学术能力,又是我国参与制定国际化通信标准的重要一步。因此,研究 ZUC 算法,具有很高的现实意义。

\section{算法流程} % 1k

ZUC 算法分成三层,如下: \cite{zuc_standard}

\begin{itemize}
    \item \textbf{线性反馈移位寄存器(Linear feedback shift register,LFSR):}LFSR 有两种模式,初始化模式和工作模式。
    \item \textbf{比特重组(Bit Reorgnization, BR):}该层将寄存器中特定位置的数据进行编排然后输出。
    \item \textbf{非线性函数(Nonlinear Function, F):}这一层涉及到一个 S 盒置换,而 S 盒置换是非线性的,因此也是这个算法中非常关键的部分。我们后续的功耗分析攻击,将会重点关注这一部分。
\end{itemize}

\vspace*{0.5\baselineskip}

ZUC 算法的运行过程如下:\cite{zuc_standard}

\begin{enumerate}
    \item \textbf{初始化阶段:}这一部分包括初始密钥的装载,LFSR 和寄存器的初始化,以及重复 32 轮的打乱操作(每一轮都包含比特重组、非线性函数以及以初始模式运行一次 LFSR)。
    \item \textbf{工作阶段:}这一部分包括一个一次性操作(比特重组、非线性函数以及以工作模式运行一次 LFSR)以及密钥生成阶段(每一次密钥输出都包含比特重组、非线性函数、输出密钥以及以工作模式运行一次 LFSR)。
\end{enumerate}

\vspace*{0.5\baselineskip}

由于密码本身相对比较复杂,尽管我已经在硬件和软件上都实现 ZUC 算法,并且验证了实现的正确性,然而想要解释清楚一些实现中具体的细节,还是太过冗长了,而且毫无必要,因此这里只是粗略地介绍了一下算法的基本组成部分和大致流程。如果想要了解关于算法的更多信息,可以参考文献 \parencite{zuc_standard}。

作为攻击者和研究人员,了解算法的全部细节和设计理由固然重要,但是更加重要的是找到算法在实际实现中可能存在的问题。设计者和攻击者考虑问题的角度是不一样的,设计者往往拥有更好的大局观,但在细节上往往无法挖掘太深,而攻击者只需要撬动整个系统中的某一点就足以达成目标了。

因此我们将在下一节讨论对 ZUC 算法实施差分功耗攻击的方法。

\vspace*{0.5\baselineskip}

图 \ref{fig:zuc_algo} 展示了 ZUC 算法的过程。

\begin{figure}[htbp]
    \centering
    \includegraphics[height=.5\textheight]{../images/zuc_algo.png}
    \caption{ZUC 算法的流程图 \cite{zuc_standard}}
    \label{fig:zuc_algo}
\end{figure}


\section{算法的软硬件实现}

硬件设计使用的软件是 ISE 14.3,使用的 FPGA 型号为:XC6SLX75-2CSG484。在完成硬件设计、仿真和综合后,在实际环境中运行 ZUC 算法,并采集功耗曲线,用于后续的分析

软件实现使用的语言为 Python 3.6,运行平台为 Windows 10。软件部分的作用是在功耗分析过程中计算中间值以及对应的假设功耗值,并进行相关系数攻击和更多的处理。软件代码开源在本人的 GitHub 仓库中:https://github.com/Hansimov/zuc-attack。

\newpage

硬件电路的输入和输出端口如图 \ref{fig:circuit_io} 所示。

\begin{figure}[htbp]
    \centering
    \includegraphics[height=.6\textheight]{../images/circuit_io.png}
    \caption{硬件电路的输入和输出端口}
    \label{fig:circuit_io}
\end{figure}

\newpage

硬件电路的内部结构如图 \ref{fig:circuit_more} 所示。

\begin{figure}[htbp]
    \centering
    \includegraphics[height=.6\textheight]{../images/circuit_more.png}
    \caption{硬件电路的内部结构}
    \label{fig:circuit_more}
\end{figure}

\newpage


\section{本章小结}

本章首先讨论了 ZUC 算法的提出背景,介绍了其和通信行业发展的密切联系,并且指出了该算法对我国的重要意义。

然后我们讨论了 ZUC 算法的流程和结构,分别介绍了 ZUC 算法的三个部分,包括线性移位反馈寄存器、比特重组和非线性函数,并且讲解了初始化和工作的运行过程。

最后我们展示了 ZUC 算法的软硬件实现情况,给出了硬件电路的原理图,开源了软件代码,并且介绍了硬件部分和软件部分各自的用途。

我们讨论了功耗分析中简单功耗分析和差分功耗分析的异同,并且重点介绍了差分功耗攻击的基本流程和一般方法。

最后我们讨论了针对 ZUC 算法的差分功耗分析方案,除了介绍了传统的差分功耗分析的流程,还依据相关资料选取了有效的中间值,并且解释了这样选取的具体原因。
