%# -*- coding: utf-8-unix -*-
%%==================================================
%% 03_zuc.tex % 2.5k
%%==================================================

\chapter{祖冲之算法}
通常的差分功耗分析都是以分组密码作为研究对象的,而本文则以祖冲之密码算法为例来说明,差分功耗分析一样可以攻击序列密码算法。

在实施攻击之前,攻击者一定要对算法本身有充分地研究,这样才能找到可能泄露信息的地方,并加以利用。这一章我们将介绍祖冲之算法的详细知识,试图从算法的流程中寻找可以攻击的位置。

\section{算法背景} % 0.5k
% 《3GPP LTE 国际加密标准 ZUC 算法》
祖冲之算法,又称 ZUC 算法,是我国提出的第一个国际商用标准密码算法。

2004 年,3GPP(3rd Generation Partnership Project,第三代合作伙伴计划)提出了 LTE(Long Term Evolution,长期演进),目的是保证该计划能够继续在电信行业拥有一定的话语权。在 2010 年底,3GPP 被确立为第四代(4G)移动通信标准。

安全性是通信技术中一个非常重要的指标,而密码算法又是保障安全性的一个重要工具。3GPP 之前已经拥有的两个算法是 AES 和 SNOW 3G,而 ZUC 算法则是第三个被纳入标准的算法。ZUC 算法的提出和设计历经了很多挑战,因为商用密码算法的要求非常严苛,既要保证极高的安全性,又要拥有较高地运行性能,还需要在各自环境下都方便实现。中国科学院等单位克服了重重困难,最终研制成功,经由中国通信标准化协会与工信部向 3GPP 组织提交了这一算法,并且经过了行业严格的评审,最终被批准成为 LTE 中的密码算法标准,参与到实际的商业应用。

总之,ZUC 算法的提出,使我国在国际商用密码领域拥有了更多的自主权,既体现了我国在密码学领域的学术能力,又是我国参与制定国际化通信标准的重要一步。因此,研究 ZUC 算法,具有很高的现实意义。
\section{算法流程} % 1k

\section{符号解释} % 0.5k
\section{实现细节} % 1k

