%# -*- coding: utf-8-unix -*-
%%==================================================
%% 00_intro.tex
%%==================================================


\chapter{绪论}

\label{chap:intro}

\section{课题背景与研究意义}

经过多年的学术研究和工业应用,密码学理论已经日趋系统和完善,各种密码算法广泛应用于各种工业设备,以保障系统和数据的安全。

目前,那些得到广泛使用的密码算法,通常都经过数学上的严格论证,并且经过了大量专家的研究和改进,因而在理论上基本是安全的。然而在现实生活中,这些算法都运行在具体的设备上,因此可能会暴露出各种各样的安全问题,研究者和攻击者可以藉由各种手段,获取密码设备中的秘密信息。

在诸多攻击密码算法和密码设备的手段中,旁路攻击是极为优秀和实用的一类。“旁路”的含义是利用实际密码设备泄露的信息,而不是利用密码算法本身的漏洞。根据时间、成本以及仪器的不同,旁路攻击又可以划分为很多种类,具体的方法也五花八门。在实际应用中,旁路攻击的效果远优于传统的密码学理论分析,因此得到了攻击者和研究人员的青睐。

在诸多旁路攻击的方法中,功耗分析是研究最多的一种,其应用也最广。功耗分析攻击所需的花费极低,即使是普通的采集设备和示波器,辅以简单的软件程序,就能完成攻击。与此同时,功耗分析攻击不会对设备进行拆解动作,因此不会留下破坏的痕迹,也不会损伤设备,这既降低了研究的成本,也减少了攻击被发现的可能性。因此,作为旁路攻击的一种最为典型的方法,我们有必要详细研究这种方法。研究清楚了功耗分析攻击,再去研究其他类型的攻击方式就要容易多了。

通常的差分功耗分析都是以分组密码作为研究对象的,而本文则以祖冲之密码算法为例来说明,差分功耗分析一样可以攻击序列密码算法。

在实施攻击之前,攻击者一定要对算法本身有充分地研究,这样才能找到可能泄露信息的地方,并加以利用。我们将介绍祖冲之算法的详细知识,试图从算法的流程中寻找可以攻击的位置。

\section{国内外研究现状}

针对 ZUC 算法的分析和攻击,国内外已经有了一些相关的研究和结果。

文献 \parencite{zuc_attack_tangming} 提出了针对 ZUC 算法的差分功耗分析攻击,选取非线性函数的输出部分,攻击出部分密钥字节,再枚举其他部分,即可实现完整的攻击。

文献 \parencite{zuc_freq} 研究了对嵌入设备上运行的 ZUC 算法实施频域攻击,其主要方法基于傅里叶变换。该研究表明,频域上泄露的信息比时域更加严重,因此攻击的效果也更好。

文献 \parencite{zuc_security} 和 \parencite{zuc_analyical} 从数学上提出了减少密钥猜测搜索空间的方法,先对密钥中的一小部分进行猜测,然后再攻击其余的部分,降低了搜索的复杂度。

文献 \parencite{zuc_iv} 和 \parencite{zuc_wu}改进了传统的功耗分析方法,通过选择初始向量来提高攻击的效率。实验表明,相比使用随机的初始向量进行差分功耗攻击,选取特定的初始向量可以大大减少所需的功耗迹数目,并且攻击的效果更加显著。

文献 \parencite{zuc_guess} 尝试了对 ZUC 和 SNOW 3G 等序列密码算法实施猜测决定攻击,将 ZUC 算法中和 32 比特相关的非线性函数拆分成了和 16 比特相关,减小了攻击时需要猜测的情况。

文献 \parencite{stream_fischer} 和 \parencite{stream_gu} 则讨论了针对更一般的序列密码算法的攻击,分析了序列密码算法中常见的组件,比如线性反馈移位寄存器,讨论了对这类组件进行攻击的可能性。

\section{本文结构}

第\ref{chap:intro}章介绍了本课题的研究背景和重要意义,探讨了国内外分析 ZUC 算法的研究现状,这些文献中提及的思路和方法对我们的实验有一定的指导和启发作用。

第\ref{chap:crypto}章介绍了现代密码学和旁路攻击的基础知识,这部分内容将让我们对整个领域有一个大致的认识,熟悉常见的密码分析方法和旁路攻击思路。

第\ref{chap:dpa}章介绍了功耗分析攻击组成部分和一般步骤,包括功耗构成、功耗仿真、功耗采集以及最经典的差分功耗分析方法。

第\ref{chap:zuc}章介绍了 ZUC 算法的背景、流程和结构,在硬件和软件上实现了这一算法,并给出了详细的攻击思路和实施方案。

第\ref{chap:results}展示了对 ZUC 算法实施差分功耗攻击的实验结果,并从不同的维度对实验结果进行了分析和解释。

第\ref{chap:conclusion}章对全文进行了总结和回顾,并且提出了可以进一步深入研究的方向。

