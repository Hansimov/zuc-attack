%# -*- coding: utf-8-unix -*-
%%==================================================
%% 00_intro.tex
%%==================================================


\chapter{绪论}

\section{课题背景与研究意义}

经过多年的学术研究和工业应用,密码学理论已经日趋系统和完善,各种密码算法广泛应用于各种工业设备,以保障系统和数据的安全。

目前,那些得到广泛使用的密码算法,通常都经过数学上的严格论证,并且经过了大量专家的研究和改进,因而在理论上基本是安全的。然而在现实生活中,这些算法都运行在具体的设备上,因此可能会暴露出各种各样的安全问题,研究者和攻击者可以藉由各种手段,获取密码设备中的秘密信息。

在诸多攻击密码算法和密码设备的手段中,旁路攻击是极为优秀和实用的一类。“旁路”的含义是利用实际密码设备泄露的信息,而不是利用密码算法本身的漏洞。根据时间、成本以及仪器的不同,旁路攻击又可以划分为很多种类,具体的方法也五花八门。在实际应用中,旁路攻击的效果远优于传统的密码学理论分析,因此得到了攻击者和研究人员的青睐。

在诸多旁路攻击的方法中,功耗分析是研究最多的一种,其应用也最广。

功耗分析攻击所需的花费极低,即使是普通的采集设备和示波器,辅以简单的软件程序,就能完成攻击。与此同时,功耗分析攻击不会对设备进行拆解动作,因此不会留下破坏的痕迹,也不会损伤设备,这既降低了研究的成本,也减少了攻击被发现的可能性。

因此,作为旁路攻击的一种最为典型的方法,我们有必要详细研究这种方法。研究清楚了功耗分析攻击,再去研究其他类型的攻击方式就要容易多了。




\section{国内外研究情况}

\section{本文结构}